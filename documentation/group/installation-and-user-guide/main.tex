\documentclass[onecolumn, draftclsnofoot,10pt, compsoc]{IEEEtran}
\usepackage{graphicx}
\usepackage{url}
\usepackage{setspace}
\usepackage{times}
\usepackage{enumitem}
\usepackage{titletoc}
\usepackage{float}
\usepackage{listings}
\usepackage{caption}
\usepackage{hyperref}
%\usepackage[export]{adjustbox}
%\usepackage[notocbib]{apacite}
\usepackage{geometry}
\geometry{textheight=9.5in, textwidth=7in}

% 1. Fill in these details
\def \CapstoneTeamName{		Beached Marine Critters Project Team}
\def \CapstoneTeamNumber{		64}
\def \GroupMemberOne{			Alea Weeks}
\def \GroupMemberTwo{			Amar Raad}
\def \GroupMemberThree{			Daniel Domme}
\def \GroupMemberFour{			Justin Disalvo}
\def \GroupMemberFive{			Zachary Tusing}
\def \CapstoneProjectName{		Develop a visual model for sea turtle beach stranding Events}
\def \CapstoneSponsorCompany{	Oregon State University Hatfield Marine Science Center; Oregon Sea Grant}
\def \CapstoneSponsorPerson{		Dr. William Hanshumaker}

% 2. Uncomment the appropriate line below so that the document type works
\def \DocType{		%Problem Statement
				%Requirements Document
				%Technology Review
				%Design Document
				%Progress Report
				Installation and User Guide
				}
			
\newcommand{\NameSigPair}[1]{\par
\makebox[2.75in][r]{#1} \hfil 	\makebox[3.25in]{\makebox[2.25in]{\hrulefill} \hfill		\makebox[.75in]{\hrulefill}}
\par\vspace{-12pt} \textit{\tiny\noindent
\makebox[2.75in]{} \hfil		\makebox[3.25in]{\makebox[2.25in][r]{Signature} \hfill	\makebox[.75in][r]{Date}}}}
% 3. If the document is not to be signed, uncomment the RENEWcommand below
\renewcommand{\NameSigPair}[1]{#1}
\doublespacing
%%%%%%%%%%%%%%%%%%%%%%%%%%%%%%%%%%%%%%%
\begin{document}
\begin{titlepage}
    \pagenumbering{gobble}
    \begin{singlespace}
     \includegraphics[height=3cm]{coe_v_spot1}
        \hfill 
        % 4. If you have a logo, use this includegraphics command to put it on the coversheet.
        %\includegraphics[height=4cm]{CompanyLogo}   
        \par\vspace{.2in}
        \centering
        \scshape{
            \huge CS Capstone \DocType \par
            {\normalsize\today}\par
            \vspace{.5in}
            \textbf{\Huge\CapstoneProjectName}\par
            %\vfill
            \vspace{1in}
            %{\Large Prepared for}\par
            %\huge \CapstoneSponsorCompany\par
            %\vspace{5pt}
            % {\Large\NameSigPair{\CapstoneSponsorPerson}\par}
            % \vspace{.5in}
            {\large Prepared by }\par
            Group\CapstoneTeamNumber\par
            % 5. comment out the line below this one if you do not wish to name your team
            %\CapstoneTeamName\par 
            \vspace{5pt}
            {\Large
                \NameSigPair{\GroupMemberOne}\par
                \NameSigPair{\GroupMemberTwo}\par
                \NameSigPair{\GroupMemberThree}\par
				\NameSigPair{\GroupMemberFour}\par
			\NameSigPair{\GroupMemberFive}\par
            }
            \vspace{20pt}
        }
        \vfill
        %\begin{abstract}
        % 6. Fill in your abstract    
        	%This document is written using one sentence per line.
        	%This allows you to have sensible diffs when you use \LaTeX with version control, as well as giving a quick visual test to see if sentences are too short/long.
        	%If you have questions, ``The Not So Short Guide to LaTeX'' is a great resource (\url{https://tobi.oetiker.ch/lshort/lshort.pdf})
		    %\noindent This document is a summary of all activities that took place during this term and the current state of the project. The outline of problems, possible solutions, and learned is also discussed. 
		    
        %\end{abstract}     
    \end{singlespace}
\end{titlepage}
\newpage
\pagenumbering{arabic}
\tableofcontents
% 7. uncomment this (if applicable). Consider adding a page break.
\listoffigures
%\listoftables
\clearpage
%\begin{singlespace}

\section{Installation}
Go to the repository: 
\newline \url{https://github.com/Tusingz/marineCritters/blob/master/Release.zip}
\newline Then download the file "Release.zip" file to your local machine. Extract the files to your desired location and run the executable "WpfApp3.exe".
\newline \begin{center}\includegraphics[width = \textwidth]{images/download-zip.PNG}\end{center}

\section{Program Features}
Once the program has started, the figure below is the screen the user will see. There is nothing on it until the user selects an option under "Map Filters" (see the section titled "Map Filters" for more information). 
\newline The user may also interact with the map with or without data displayed on it (see the section titled "Interactive Map" for more information). 
\newline \begin{center}\includegraphics[width=0.5 \textwidth]{images/startup-program.PNG}\end{center}

\subsection{Interactive Map}
Holding left click and moving the mouse will control where the user moves on the map. Using the scroll wheel will zoom in and out towards where the cursor is.

\subsection{Map Filters}
Under "Map Filters", there is four separate options to select. The user may select zero to all of the filters at a time. The data that will be displayed will only cover part of the Pacific Ocean with all of Oregon covered. 
\newline The "Turtle Data" option will automatically pull up all the turtle data, within the selected date range, for all species. To filter the species, see the section "Species Selection". 
\newline \begin{center}\includegraphics[width=0.5 \textwidth]{images/turtle-data.PNG}\end{center}
The "Wind Vectors" option will display arrow vectors on the map to indicate the wind direction. The larger and darker the arrow, the stronger the wind. 
\newline \begin{center}\includegraphics[width=0.5 \textwidth]{images/wind-vector.PNG}\end{center}
The option "Air Temperature" is a series of blue dots that indicate the temperature of an area over the ocean. Darker the blue, the warmer the area is. 
\newline \begin{center}\includegraphics[width=0.5 \textwidth]{images/air-temp.PNG}\end{center}
The last option, "Sea Surface Temperature" is the temperature at a specific location on the ocean represented by red to blue dots. The darkest blue is the coldest while the darkest red is the hottest areas. 
\newline \begin{center}\includegraphics[width=0.5 \textwidth]{images/sea-surf-temp.PNG}\end{center}

\subsection{Date Range Selector}
The data range selector is located directly under "Map Filters". The user will select a start date then an end date. The map will display the data from in between the selected dates. To apply the selected date change, click the button "Change Date Range".
\newline \begin{center}\includegraphics[width=0.5 \textwidth]{images/date-change.PNG}\end{center}

\subsection{Species Selection}
The Species Selection section filters what turtle data will be displayed on the map based on a specific spices. Zero to all may be selected at a time, and all filters will be selected by default. 
\newline \begin{center}\includegraphics[width=0.5 \textwidth]{images/turtle-data-filter.PNG}\end{center}


\subsection{Turtle Data}
The user can see the turtle data that is being displayed by selecting "Show Turtle Data Table" at the bottom of the overlay. Clicking the same button, renamed to "Hide Turtle Data Table" will hide the table again. 
\newline \begin{center}\includegraphics[width=0.5 \textwidth]{images/data-table.PNG}\end{center}

% bibliography
\pagebreak

\end{document}