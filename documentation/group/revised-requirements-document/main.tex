\documentclass[onecolumn, draftclsnofoot,10pt, compsoc]{IEEEtran}
\usepackage{graphicx}
\usepackage{url}
\usepackage{setspace}
\usepackage{times}
\usepackage{enumitem}
\usepackage{geometry}
\usepackage{float}
\geometry{textheight=9.5in, textwidth=7in}

% 1. Fill in these details
\def \CapstoneTeamName{		Beached Marine Critters Project Team}
\def \CapstoneTeamNumber{		64}
\def \GroupMemberOne{			Alea Weeks}
\def \GroupMemberTwo{			Amar Raad}
\def \GroupMemberThree{			Daniel Domme}
\def \GroupMemberFour{			Justin Disalvo}
\def \GroupMemberFive{			Zachary Tusing}
\def \CapstoneProjectName{		Develop a visual model for sea turtle beach stranding Events}
\def \CapstoneSponsorCompany{	Oregon State University Hatfield Marine Science Center; Oregon Sea Grant}
\def \CapstoneSponsorPerson{		Dr. William Hanshumaker}

% 2. Uncomment the appropriate line below so that the document type works
\def \DocType{		%Problem Statement
				Revised Requirements Document
				%Technology Review
				%Design Document
				%Progress Report
				}
			
\newcommand{\NameSigPair}[1]{\par
\makebox[2.75in][r]{#1} \hfil 	\makebox[3.25in]{\makebox[2.25in]{\hrulefill} \hfill		\makebox[.75in]{\hrulefill}}
\par\vspace{-12pt} \textit{\tiny\noindent
\makebox[2.75in]{} \hfil		\makebox[3.25in]{\makebox[2.25in][r]{Signature} \hfill	\makebox[.75in][r]{Date}}}}
% 3. If the document is not to be signed, uncomment the RENEWcommand below
%\renewcommand{\NameSigPair}[1]{#1}

%%%%%%%%%%%%%%%%%%%%%%%%%%%%%%%%%%%%%%%
\begin{document}
\begin{titlepage}
    \pagenumbering{gobble}
    \begin{singlespace}
     \includegraphics[height=3cm]{coe_v_spot1}
        \hfill 
        % 4. If you have a logo, use this includegraphics command to put it on the coversheet.
        %\includegraphics[height=4cm]{CompanyLogo}   
        \par\vspace{.2in}
        \centering
        \scshape{
            \huge CS Capstone \DocType \par
            {\normalsize\today}\par
            \vspace{.5in}
            \textbf{\Huge\CapstoneProjectName}\par
            %\vfill
            \vspace{1in}
            {\Large Prepared for}\par
            \huge \CapstoneSponsorCompany\par
            \vspace{5pt}
            {\Large\NameSigPair{\CapstoneSponsorPerson}\par}
            \vspace{.5in}
            {\large Prepared by }\par
            Group\CapstoneTeamNumber\par
            % 5. comment out the line below this one if you do not wish to name your team
            %\CapstoneTeamName\par 
            \vspace{5pt}
            {\Large
                \GroupMemberOne\par
                \GroupMemberTwo\par
                \GroupMemberThree\par
				\GroupMemberFour\par
			\GroupMemberFive\par
            }
            \vspace{20pt}
        }
        \vfill
        \begin{abstract}
        % 6. Fill in your abstract    
        	%This document is written using one sentence per line.
        	%This allows you to have sensible diffs when you use \LaTeX with version control, as well as giving a quick visual test to see if sentences are too short/long.
        	%If you have questions, ``The Not So Short Guide to LaTeX'' is a great resource (\url{https://tobi.oetiker.ch/lshort/lshort.pdf})
		    The Pacific Ocean off of the Oregon Coast has constantly changing weather and sea conditions.  With this, many different animal species, including threatened sea turtles, end up stranded
            on the shore and perishing. In order to better understand how weather and ocean conditions affect where and when animals get stranded, historical statistics will need to be combined and reviewed. This will aid in the finding of correlations to help rescuers to find stranded animals in a timely manner. This document outlines the project program that will help to accomplish these tasks.  The minimum requirements are outlined in detail.
        \end{abstract}     
    \end{singlespace}
\end{titlepage}
\newpage
\pagenumbering{arabic}
\tableofcontents
% 7. uncomment this (if applicable). Consider adding a page break.
%\listoffigures
\listoftables
\clearpage

% 8. now you write!
\begin{singlespace}

\section{Revision Table}
\begin{table}[H]
\begin{center}
 \begin{tabular}{ |p{0.1\textwidth}|p{0.3\textwidth}|p{0.3\textwidth}|} 
 \hline
 \multicolumn{1}{|c|}{\textbf{Section}} 
 & 
\multicolumn{1}{|c|}{\textbf{Old Requirement}}  & 
\multicolumn{1}{|c|}{\textbf{Revision}}\\
 \hline
 
4.2 & The user should be able to download the software program from the client's website. &  We do not currently have access to our client's website server or source code. For this reason we cannot complete this requirement. \\
 \hline
 
 4.2 & The user should be able to import data files that contain data on stranded animals. & Our standalone project executable is not a powerful enough engine to process data. Instead of uploading data to our program, it would be better to upload data to a more powerful tool like ArcGIS Pro. ArcGIS Pro is designed to handle data processing.\\
 \hline
 
  4.2 & The user should be able to display imported data in a table. & This requirement cannot be completed because it is dependent on the above requirement.\\
 \hline


\end{tabular}
\caption{Table 1: Document Revisions}
\label{table:1}
\end{center}
\end{table}


\section{Introduction}
This section provides information on the contents for the rest of the document.  Sources and definitions are also provided.
\subsection{Purpose}
The purpose of this document is to serve as a guide to the creation of a visual representation of beached sea turtles. A detailed description of the project and related research is outlined. \newline 
\par \noindent The audience for this document is anyone who is involved in the development of the aforementioned model.
\subsection{Scope}
    These requirements are meant to help guide the project team in the meeting of the client's, Dr. William Hanshumaker, needs and agreed upon functions.  This will help form correlations about how ocean and weather conditions affect where sea turtles become stranded.
\subsection{Definitions}
    \textbf{Geographic Information System (GIS)} - A program designed to process and represent multiple forms of dataand layer them over maps [1]. \newline \newline
    \textbf{Olive Ridley Sea Turtle} - A species of sea turtle that is mainly found in warm climates. This species is the primary focus for this project [2]. \newline \newline
    \textbf{NOAA} - An abbreviation for Nation Oceanic and Atmospheric Administration. This administration provides scientific data about weather and sea conditions [3]. \newline \newline
    \textbf{NANOOS} - An abbreviation for Northwest Association of Networked Ocean Observation Systems.  This association monitors and studies sea levels and sea life through data collection and visualizations [4].
    %\begin{itemize}
    %    \item Geographic Information System (GIS) - A program designed to process and represent multiple forms of data
    %        and layer them over maps [1].
    %    \item Olive Ridley Sea Turtle - A species of sea turtle that is mainly found in warm climates. This species is the primary focus for this project because it gets stranded on Oregon shores [2].
    %    \item NOAA - An abbreviation for Nation Oceanic and Atmospheric Administration. This administration provides scientific data about weather and sea conditions [3].
    %    \item NANOOS - An abbreviation for Northwest Association of Networked Ocean Observation Systems.  This association monitors and studies sea levels and sea life through data collection and visualizations [4].
    %    \item ArcGIS - A GIS software provider that enables using geographic maps to analyze data [5].
    %\end{itemize}
\subsection{References}
The following resources will be used to collect data and provide assistance to the project:
\begin{enumerate}[label={[\arabic*]}]
    \item A. Lovett and K. Appleton, \textit{GIS for Environmental Decision-Making}. Boca Raton, FL: CRC Press, 2008. [E-book]. Available: ProQuest Ebook Central.
    \item National Oceanic and Atmospheric Administration Fisheries, "Olive Ridley Turtles," \textit{National Oceanic and Atmospheric Administration}. [Online]. Available: \url{https://www.fisheries.noaa.gov/species/olive-ridley-turtle}. [Accessed: Oct. 28, 2018].
    \item National Oceanic and Atmospheric Administration, "National Oceanic and Atmospheric Administration," \textit{Noaa.gov}. [Online]. Availabe: \url{https://www.noaa.gov/}. [Accessed: Oct. 28, 2018].
    \item Northwest Association of Networked Ocean Observation Systems (NANOOS), "NANOOS," \textit{NANOOS}. [Online]. Available: \url{http://www.nanoos.org/home.php}. [Accessed: Oct. 28, 2018].
    \item ArcGIS, "ArcGIS | Main," \textit{arcgis.com}. [Online]. Available: \url{https://www.arcgis.com/index.html}. [Accessed: Oct.28, 2018].
    \item Marine Mammal Institute, "Oregon Marine Mammal Stranding Network," \textit{Oregon State University Marine Mammal Institute}. [Online]. Available: \url{https://mmi.oregonstate.edu/ommsn}. [Accessed: Oct. 28, 2018].
    \item Seaside Aquarium, "SEASIDE AQUARIUM." [Online]. Available: \url{http://www.seasideaquarium.com/}. [Accessed: Oct. 28, 2018].
    \item NANOOS Visualization System (NVS), "NVS: Tuna Fishers," \textit{NANOOS Visualization System}. [Online]. Available: \url{http://nvs.nanoos.org/TunaFish}. [Accessed: Oct. 28, 2018].
    \item Oregon Shores Conservation Coalition, "Coastwatch | Oregon Shores," \textit{Oregon Shores}. [Online]. Available: \url{https://oregonshores.org/coastwatch}. [Accessed: Oct. 28, 2018].
    \item U.S. Fish and Wildlife Service, "Oregon Fish and Wildlife Office - Newport Field Office," \textit{Fws.gov}. [Online]. Available: \url{https://www.fws.gov/oregonfwo/promo.cfm?id=177175715}. [Accessed: Oct. 28, 2018].
\end{enumerate}
    Additionally, various research papers will be read throughout the duration of the project.
\subsection{Overview}
    The requirements document is intended to provide clear guidelines and project details that will assist in making data visualizations that include weather and oceanographic conditions and stranded animals.
    
    \subsubsection{Section 2}
    Section 2 provides functional details on the project product and its intended users.
    
    \subsubsection{Section 3}
    Section 3 displays all specific requirements and goals of the project.  This includes any stretch goals. A chart is provided to show a time line for finishing product features.
    %\end{itemize}
    %These guidelines are contained in the following sections:
    %\begin{itemize}
    %    \item Section 2: This section provides functional details on the project product and its intended users.
    %    \item Section 3: This section displays all specific requirements and goals of the project.  This includes any stretch goals. A chart is provided to show a time line for finishing product features.
    %\end{itemize}
\section{Overview Description}
The purpose of this section is to provide a general overview of the visual model for sea turtle beach strandings software. All major product functions and user characteristics are explained, and any constraints and assumptions will be outlined.
\subsection{Product Perspective}
This product allows researchers and rescue teams to look at data in a visual way. Without the product there are no visual constructions of data surrounding marine life strandings.
\subsection{Product Functions}
The visual model of stranded sea turtles will incorporate the following data.
    \subsubsection{Data on turtle strandings}
    The location where the turtle was found and the date the turtle was found.
    \subsubsection{Data on climate and weather}
    Information on the wind current and direction and any unusual, or annual, events will be taken into consideration.
    \subsubsection{Data on ocean conditions}
    This includes data of sea surface temperatures and ocean currents.
    %\begin{itemize}
    %    \item Basic data about the turtle stranding:
    %    \begin{itemize}
    %        \item The location of where the turtle was found
    %        \item The date the turtle was found        
    %    \end{itemize}
    %    \item Climate and weather data from the day the turtle was beached:
    %    \begin{itemize}
    %        \item Wind Current 
    %        \item Wind Direction
    %        \item El Nino or La Nina year
    %    \end{itemize}
    %    \item Ocean conditions:
    %    \begin{itemize}
    %        \item Sea surface temperatures
    %        \item Ocean currents
    %    \end{itemize}
    %\end{itemize}
\subsection{User Characteristics}
    The intended users of this software are animal rescue workers and researchers.  
    \begin{itemize}
        \item Researcher: marine biologists will be able to examine trends and statistics in the archived data and possibly be able to find correlations.  The data might also be the basis for future research. 
    \end{itemize} 
\subsection{Constraints}
The team would prefer to use ArcGIS due to the features it provides. 
\subsection{Assumptions and Dependencies}
The success of this project relies on the availability of data on sea turtle beach strandings. The planning of this project is done with the assumption that the team will be able to gather enough data on sea turtle beachings to create a meaningful model. If there is not enough data, then our visual representation of the data will serve little purpose. \newline 
Another dependency is that the data collected may be organized in various forms so it must be possible to standardize the data received. 
\section{Specific Requirements}
This section of the document will outline all interface and functional requirements, along with projected stretch goals and a projected time line of project milestones.
\subsection{Interface}
\subsubsection{User Interface}The interface implemented in this project will be an executable download from the client's web page. In the executable, the user will be able to select what data they want included on a map, such as weather and water surface temperature. The user will also be able to choose the date or date range that they want to view.  Then, a map will be generated with locations of where an animal has been found stranded on the beach.  The overlays of the map will be dynamic based on what data the user wants to display with regards to weather, sea conditions, and date.
\subsubsection{Software Interface}
In the background, the software will need to take in a database or list of 
sea animal strandings. This software will allow for the user to input information about each individual animal's strandings so that the GUI will provide a map overlay. This sort of database connection will allow users to quickly and efficiently access individual animals information and visual overlays.
\subsection{Functional Requirements}
    This section describes various use cases for our project. \newline\newline
    \textbf{ID:FR1}\newline
    TITLE: Program Download\newline
    DESCRIPTION:The user should be able to download the software program from the client's website.\newline
    RATIONALITY:This allows the user to access the functions of the software.\newline
    DEPENDENCIES:None\newline
    
    \textbf{ID:FR2}\newline
    TITLE: Open Software\newline
    DESCRIPTION:The user should be able to start the software to start using the map.\newline
    RATIONALITY:This allows the user to start with a blank map.\newline
    DEPENDENCIES:FR1\newline

    \textbf{ID:FR3}\newline
    TITLE: Import Data\newline
    DESCRIPTION:The user should be able to import data files that contain data on stranded animals.\newline
    RATIONALITY:This allows the user to start to manipulate the chosen data on the map and in the tables.\newline
    DEPENDENCIES:FR2\newline

    \textbf{ID:FR4}\newline
    TITLE: Display Map\newline
    DESCRIPTION:The map will be visible to the user.\newline
    RATIONALITY:This allows the user to get a visual representation of data based on location and conditions.\newline
    DEPENDENCIES:FR2\newline

    \textbf{ID:FR5}\newline
    TITLE: Interactive Map\newline
    DESCRIPTION:The user should be able to manipulate the map with the mouse.  The user should be able to zoom and change location of view.\newline
    RATIONALITY:This allows for the user to get the visualization of data that is desired, and it allows for a simplified interface to the map.\newline
    DEPENDENCIES:FR4\newline

    \textbf{ID:FR6}\newline
    TITLE: Map Overlays\newline
    DESCRIPTION:The user should be able to change overlays of the map.  These overlays should include sea surface temperature, sea currents, and wind direction.\newline
    RATIONALITY:This allows the user to examine the effect conditions have on the timing of animals becoming stranded.\newline
    DEPENDENCIES:FR4\newline

   \textbf{ID:FR7}\newline
    TITLE: Map Date\newline
    DESCRIPTION:The user should be able to view weather and ocean conditions based on the desired date.\newline
    RATIONALITY:This allows the user to look for trends in conditions and animal behavior. \newline
    DEPENDENCIES:FR4\newline

    \textbf{ID:FR8}\newline
    TITLE: Map Data Point Inclusion\newline
    DESCRIPTION:The user should be able to include imported data points for stranded animals.\newline
    RATIONALITY:This will allow the user to visualize stranded animals on the map in relation to the weather and beach location.\newline
    DEPENDENCIES:FR3, FR4\newline

    \textbf{ID:FR9}\newline
    TITLE: Map Data Point Selection Criteria\newline
    DESCRIPTION:The user should be able to decide which data points to include on the map based on animal species, date ranges, and location.\newline
    RATIONALITY:This will help the user to start examine trends with species in relation to weather and ocean conditions.\newline
    DEPENDENCIES:FR8\newline

    \textbf{ID:FR10}\newline
    TITLE: Data Tables Displayed\newline
    DESCRIPTION:The user should be able to display imported data in a table.\newline
    RATIONALITY:This will give the user an easy method to look at the data without having to click through the map.\newline
    DEPENDENCIES:FR3\newline

    \textbf{ID:FR11}\newline
    TITLE: Data Table Sort\newline
    DESCRIPTION:The user should be able to sort the data table based on a chosen criteria.  This criteria should include date, species, location, animal outcome, and animal age.\newline
    RATIONALITY:This will allow the user to organize the data in a way that is desired.\newline
    DEPENDENCIES:FR10\newline

    \textbf{ID:FR12}\newline
    TITLE: Data Error Handling\newline
    DESCRIPTION:When the user imports data that contains an error, the user should be notified with a description of the problem and a possible solution.\newline
    RATIONALITY:This gives the user feedback on data importation issues and how to ameliorate the problems.\newline
    DEPENDENCIES:FR3\newline
    
\subsection{Performance Requirements}
This section outlines the minimum performance requirements that the project must achieve for the client.\newline\newline
    \textbf{ID:QR1}\newline
    TITLE: Easy Installation\newline
    DESCRIPTION:The software download package should include all necessary software files to run the program after it has been installed.  There should be different downloads for different computer operating systems.\newline
    RATIONALITY:This allows the user to use the software without any unnecessary steps.\newline
    DEPENDENCIES:None\newline

    \textbf{ID:QR2}\newline
    TITLE: Intuitive Map and Data Manipulation\newline
    DESCRIPTION:Manipulation of the map and data should be easy to understand, and the user interface should perform how a user has come to expect based on popular software.\newline
    RATIONALITY:This lowers the burden on the user to learn how to use the software, which would detract from functionality.\newline
    DEPENDENCIES:None\newline

    \textbf{ID:QR3}\newline
    TITLE: Intuitive Menu\newline
    DESCRIPTION:The menu and main functions of the software should be clear, visible, and easy to find.\newline
    RATIONALITY:This will enable the user to use the software with less time wasted on completing simple tasks.\newline
    DEPENDENCIES:None\newline

    %\textbf{ID:QR4}\newline
    %TITLE: Program Responsiveness\newline
    %DESCRIPTION:\newline
    %RATIONALITY:\newline
    %DEPENDENCIES:None\newline
    
    %\begin{itemize}
    %    \item Use a preexisting database with the species of animal, date found, location found, and weather/sea conditions during that period of time
    %    \item Use the database to map a specific animal over a period of time along with weather and sea conditions
    %    \item Record inputted data to the main database
    %\end{itemize}
    
    %\begin{itemize}
    %    \item Generalize the program so any species of animal can be mapped
    %    \item Implement a predictive model
    %    \item Write a research paper
    %    \item Draw correlations between global warming and animal migration and available food sources
    %    \item Learn more about El Nino and La Nina climate cycles
    %    \item Learn more about the life cycles of various sea creatures
    %    \item Conclude various suspects and sources to the cause of a sea critter's stranding
    %\end{itemize}

\subsection{Gantt Chart}
\includegraphics[scale=0.5,angle=270,origin=c]{GanttChart-Capstone.eps}
\end{singlespace}
\end{document}