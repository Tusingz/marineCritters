\documentclass[12pt]{article}
%\usepackage{times}
\usepackage[utf8]{inputenc}
 
\usepackage{tgbonum}

\usepackage{cite}
\usepackage{graphicx}
\usepackage{url}
\setlength{\parskip}{1em}
\setlength{\parindent}{0em}
%this is a comment
\fontfamily{cmss}
\selectfont


\title{CS 461 \\ Problem Statement: \\ Beached Marine Critters \\}
\author{Amar Raad (raadv)}
\date{October 11th, 2018 \\ CS 461 | Fall 2018}

\begin{document}
\maketitle
\section{Abstract}
The Beached Marine Critters Application develops a predictive model for beach standings of sea turtles 
and potentially other related marine creatures. 
Although currently functioning decently, enhancements and improvements can be implemented to obtain more accurate results. 
The team of CS students will implement new features and variables to help obtain and predict more accurate stranding patterns of marine critters. The end result will be a much stronger and reliable application to be the successor of the application's previous version.

\newpage
\tableofcontents
\newpage
\section{Definition and Description}
Tracking the species and locations of stranded marine animals is important for understanding many aspects of our coastal environment. With the help of relevant data and coordinates, the stranding patterns of various species can then be identified. It can then be possible for predictive models to be built to try identifying where such marine life will strand and how these future patterns may change. The data collected and displayed also can be used for researching various topics such as climate change and migrations.

When recording any new information to the database and records, the coordinates and species data should be updated with the relevant information, to increase the efficiency and calculate more accurate pattern predictions.

\section{Proposed solution}
The Application currently develops a GIS with sea surface temperature by using an existing data base of stranded animal locations and dates to, wind and current direction. Such information is used to develop a predictive model for different beach strandings.

To introduce a new solution for this project, more variable and parameters can be taken into account to calculate even more accurate results. Various factors such as the Tide comparison to strandings with visual representation can help in generating more possible stranding predictions. Other features and options such as overwriting with updated sea creature stranding statistics and records, as well as recognizing inaccurate or false recordings. Improvements to the UI and location tracking features can be implemented as well.

\section{Performance metrics}
Divided among a team of five, the program application will be improved, enhanced, and further expanded.
To meet milestone goals, there will be a schedule followed by the team to complete given assignments within the turnaround time of each term. By each milestone, the featured goal will behave and perform as a working prototype of the specifically desired functionality.

Time and room for error and unexpected occurrences should be taken into account, allowing a cushion to accommodate delays in the application's progress. 
 

\end{document}
