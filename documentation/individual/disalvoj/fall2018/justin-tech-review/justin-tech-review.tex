\documentclass[onecolumn, draftclsnofoot,10pt, compsoc]{IEEEtran}
\usepackage{graphicx}
\usepackage{url}
\usepackage{setspace}
\usepackage{times}
\usepackage{enumitem}
\usepackage{geometry}
\geometry{textheight=9.5in, textwidth=7in}

% 1. Fill in these details
\def \CapstoneTeamName{   Beached Marine Critters Project Team}
\def \CapstoneTeamNumber{   64}
\def \GroupMemberOne{     Alea Weeks}
\def \GroupMemberTwo{     Amar Raad}
\def \GroupMemberThree{     Daniel Domme}
\def \GroupMemberFour{      Justin Disalvo}
\def \GroupMemberFive{      Zachary Tusing}
\def \CapstoneProjectName{    Develop a visual model for sea turtle beach strandings}
\def \CapstoneSponsorCompany{ Oregon State University Hatfield Marine Science Center; Oregon Sea Grant}
\def \CapstoneSponsorPerson{    Dr. William Hanshumaker}

% 2. Uncomment the appropriate line below so that the document type works
\def \DocType{    %Problem Statement
        %Requirements Document
        Technology Review
        %Design Document
        %Progress Report
        }
      
\newcommand{\NameSigPair}[1]{\par
\makebox[2.75in][r]{#1} \hfil   \makebox[3.25in]{\makebox[2.25in]{\hrulefill} \hfill    \makebox[.75in]{\hrulefill}}
\par\vspace{-12pt} \textit{\tiny\noindent
\makebox[2.75in]{} \hfil    \makebox[3.25in]{\makebox[2.25in][r]{Signature} \hfill  \makebox[.75in][r]{Date}}}}
% 3. If the document is not to be signed, uncomment the RENEWcommand below
\renewcommand{\NameSigPair}[1]{#1}

%%%%%%%%%%%%%%%%%%%%%%%%%%%%%%%%%%%%%%%
\begin{document}
\begin{titlepage}
    \pagenumbering{gobble}
    \begin{singlespace}
     \includegraphics[height=3cm]{coe_v_spot1}
        \hfill 
        % 4. If you have a logo, use this includegraphics command to put it on the coversheet.
        %\includegraphics[height=4cm]{CompanyLogo}   
        \par\vspace{.2in}
        \centering
        \scshape{
            \huge CS Capstone \DocType \par
            {\normalsize\today}\par
            \vspace{.5in}
            \textbf{\Huge\CapstoneProjectName}\par
            %\vfill
            \vspace{1in}
            {\Large Prepared for}\par
            \huge \CapstoneSponsorCompany\par
            \vspace{5pt}
            {\Large\NameSigPair{\CapstoneSponsorPerson}\par}
            \vspace{.5in}
            {\large Prepared by }\par
            Group\CapstoneTeamNumber\par
            % 5. comment out the line below this one if you do not wish to name your team
            %\CapstoneTeamName\par 
            \vspace{5pt}
            {\Large
                %\NameSigPair{\GroupMemberOne}\par
                %\NameSigPair{\GroupMemberTwo}\par
                %\NameSigPair{\GroupMemberThree}\par
        \NameSigPair{\GroupMemberFour}\par
          %\NameSigPair{\GroupMemberFive}\par
            }
            \vspace{20pt}
        }
        \vfill
        \begin{abstract}
        % 6. Fill in your abstract    
          %This document is written using one sentence per line.
          %This allows you to have sensible diffs when you use \LaTeX with version control, as well as giving a quick visual test to see if sentences are too short/long.
          %If you have questions, ``The Not So Short Guide to LaTeX'' is a great resource (\url{https://tobi.oetiker.ch/lshort/lshort.pdf})
        \noindent The Pacific Ocean off of the Oregon Coast has constantly changing weather and sea conditions.  With this, many different animal species, including threatened sea turtles, end up stranded on the shore and perishing. In order to better understand how weather and ocean conditions affect where and when animals get stranded, historical statistics will need to be combined and reviewed. The tools needed to do this will be important. This document reviews a few technologies that could be used for the project. The development environment, how the data will be stored and organized, and the portability of the program are three of the many categories that will be reviewed. 
        \end{abstract}     
    \end{singlespace}
\end{titlepage}

\newpage
\pagenumbering{arabic}

\tableofcontents
\newpage

\section{Introduction}
Data regarding sea turtle standings is far and few between. Taking that into consideration, determining the development environment for the program, the way data will be stored and standardized, and the portability of the project will be important factors to consider. When determining the development environment, the team will use git to keep track of progress and changes within the project. The programs used in developing the program could be Python, JavaScript, or C++. Standardizing and storing data from multiple sources will be done in MySQL, excel, or Oracle. Taking into consideration who will be using the program and where it will be used will effect the out come for portability; this can be a downlaodable program, a web application, or a phone application. However, before considering how the data will be stored or the type of program the group will create, the development environment will be decided first. 

\section{Development Environment}
As stated, git will be used as the version control system used between the members of the group. This allows each user to work from their own PC without the need to be in the same building or room. The programming language to implement the project will likely be Python due to the type of the project and the tools available. One of the main tools needed is a geographic information system (GIS).
    \subsection{Python}
    Python is an interpreted, object-oriented, high-level programming language. It's used for rapid app development, scripting, and connecting existing components of an application together. The debugging tool is fast and easy to use and often times will return easy to understand messages\cite{pythonwebsite}. Python is easy to use and read, making it ideal for small team of programmers. However, some negative aspects of python is that it's very slow. It is not ideal for memory intensive tasks and it has limitations with database access. It's important to note that python has no issue using MySQL or reading from files like excel. However, Oracle is primarily used with Java, therefore they are not compatible\cite{pythonproscons}. A large factor in deciding what language to use will be how well it can generate a GIS. Python has many libraries for this function, one of them being ArcGIS\cite{pythongis}.
    
    \subsection{JavaScript}
    JavaScript is a scripting programming language used to implement complex things on web page. JavaScript is often used on op of HTML and CSS in web application development\cite{javascriptsite}. Therefore, if the project will be a web app, then JavaScript will be the main tool of implementing the program. JavaScript can be connected with a MySQL database very easily, however, it is not the best solution when wanting to read information from a file. Additionally, JavaScript can be interpreted differently depending on what browser the user is on. This could cause some unwanted and unknown bugs with different versions of web browsers and increase testing time. JavaScript can create a GIS, but, it is not the best choice. Using a non-scripting language like Python is much easier when creating something like a GIS. The most popular GIS for JavaScript is ESRI API\cite{javascriptgis}.
    
    \subsection{C++}
    C++ is a general-purpose object-orientated language and is an extension of C. It is most commonly used for comparisons, arithmetic, bit manipulation and logical operators. C++ allows users with a lot of freedom to implement a large variety of projects\cite{cppsite}. Some pros of C++ is that it has very wide support due to its popularity. Also, it is very powerful and fast. It has very similar syntax as other programming languages like C and Java making it easy for programmers to switch between the different languages. However, some cons of C++ is that due to it being so powerful, it can be unsafe to use because of how much freedom the user is given. Additionally, there is little memory management built in to C++ and it can be complicated to use and read\cite{cppproscons}. There is one C++ library available: MITAB. However, there is a free, no licensing restriction, C program called ShapeLib which is used for GIS\cite{cppgis}.
    
\section{Data Storage \& Standardization}
Data will come from many different sources with their own naming and storage conventions. One task is to get all the data needed, standardize the data, and store the data. When standardizing, the name of the species, location found, date found, weather conditions around the found time, and sea temperature around the found time are the main pieces of information needed. Therefore, a method for storing the data into a database or file for easy access is essential. 
    \subsection{MySQL}
    SQL is a query language used for managing a database. There are many branches of SQL that have slightly different purposes. The most likely branch of SQL that would be used in this project is MySQL. MySQL is relatively easy to develop and highly portable\cite{sqlsite}. In terms of this project, the data needs to be organized in a standardized way that data from multiple sources will be able to work together easily. At first, the database will be for Sea Turtles, however, if the stretch goal of generalizing the project to include many species of fish is met, then organizing it in such a way is important. Therefore, the data should be a list of the different fish and animals of interest, at first only sea turtles. Each animal will have specific date found, location found, weather conditions around the time and place found, and sea conditions around the time and place found.
    
    \subsection{Excel}
    It is possible to use Microsoft Excel as a database. Excel is very popular among many business and many people have it on their PC at home. This creates a very highly portable database. However, by using excel as a database, the user loses many aspects advantages of using a different option, like MySQL. Some cons of using excel as a database is a lack of security that comes with many other databases. Also, as the database gets larger, the size of the file and performance reading form it will suffer\cite{excelsite}. However, another option of using excel as a database is creating a .csv file from excel and using that via file manipulation. A CSV file is organized by commas and new line characters making predicting the outcome of a file very simple. However, if this method were to be implemented, very strict rules on how the document is formatted must be followed. These rules would be included with the program and file.
    
    \subsection{Oracle}
    Oracle is a database where all data is threaded as a unit. The purpose of Oracle is to store and retrieve information the user wants. Also, Oracle is the first database designed for enterprise grid computing. Enterprise grid computing creates large pools of industry-standard storage and servers. Oracle's purpose is for large scale project or businesses to keep their data safe and in order. In terms of this project, the database would be used to a much smaller scale than it is designed for. However, if the project grows to many different species of fish, there may be enough data to justify the use of a power database system like this. Another factor to consider is how safe does the data need to be. The most important thing is preventing users from tampering with it and corrupting the data. Therefore, a database that will prevent that is very important. Additionally, preventing unauthorized users from viewing, adding, or altering any data is something to consider when choosing the right database. Oracle can be used in developing applications with C, C++, Java, and Visual Basic\cite{oraclesite}.
    
\section{Portability}
When choosing the type of implementation method, the portability of that method is a factor. Creating a downloadable program is very portable but requires factors like type of operating system. A web app is more complicated but only requires internet connection. A phone app can be taken anywhere but will require a lot of extra work.
    \subsection{Downloadable Program}
    Creating a downloadable link on the clients already existing website is one method to consider when thinking about portability. The benefit of creating a download link to the program is that it will be portable to any computer. The computer must be connected to the internet at first, but once the application is downloaded, the user can be offline when using the program. This will allow a user to go directly to a source of a situation, outside of Wi-Fi, and still be able to use the program. Keeping in mind the size of the program, the download may download a .zip file rather than the full program\cite{downloadsite}. Additionally, when creating a downloadable program, the operating system it will be ran off of needs to be considered. 
    
    \subsection{Web App}
    Creating a web app that can be put on the clients website will increase ease of use and portability. However, the user must be connected to the internet at all times to use it. If that may be an issue is something the group needs to discuss with the client and his input may effect the outcome. Some issues with having a completely web based application is the complexity of programming it compared to creating an application in Python or another programming language. A separate web page would be made and the HTML, CSS, and JavaScript to make it look nice and running smoothly may be more work that it's worth. Additionally, the issue of making sure every web browser is compatible with the application could cause some problems. 
    
    \subsection{Phone App}
    Utilizing a phone app is a possibility, however, there are far more cons than there are pros. The only way a phone app will happen is if our client wants a mobile app. The portability of a phone app would be ideal, however, unless the database was on the users phone, the user must be connected to the internet somehow. This is similar to the web application, however the user cannot access this on their desktop or laptop without an emulator, which if that was the case, using a downloaded program or a web application would be the far simpler and faster option to take. 
    
\section{Conclusion}
To complete this project, many factors of how the project will be implemented must be thoroughly inspected and thought over. After looking into a few different development environment, data storage methods, and which options were portable and practical, it is most likely that a downloadable program ran off of python will be the starting point the group will take. If the client prefers, a web-based application would be the next best solution. 

\newpage

% bibliography
\nocite{*}%if nothing is referenced it will still show up in refs
\bibliographystyle{IEEEtran}
\bibliography{justin-tech-review}
%end bibliography

\end{document}
