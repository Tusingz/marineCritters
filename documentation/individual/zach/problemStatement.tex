\documentclass[onecolumn, draftclsnofoot,10pt, compsoc]{IEEEtran}
\usepackage{graphicx}
\usepackage{url}
\usepackage{setspace}

\usepackage{geometry}
\geometry{textheight=9.5in, textwidth=7in}

% 1. Fill in these details
\def \ClassName{        CS 461: Senior Capstone }
\def \GroupMemberThree{         Zachary Tusing }


            

%%%%%%%%%%%%%%%%%%%%%%%%%%%%%%%%%%%%%%%
\begin{document}
    \begin{center}
    \huge\bf{ } 
   
    \large\textbf{\textit{Problem Statement}}\par
     
    
    
    \small{\bf\textit \ClassName Oregon State University}\par
    
    
    {\bf\textit{ Fall Term, October 11th, 2018} }
    
    
   {\small {\bf Written by:} \GroupMemberThree }
        \end{center}
    \begin{abstract}
    This document provides an explanation of the problem and proposed solution. The outlined problem is that rescue teams are unable to track down enough sea creatures. The proposed solution is to create a predictive model for beach stranding of sea turtles, sharks, squids, and other marine life. This will be done using a geographic information system(GIS) to provide a map of where animals were stranded or may be stranded. This GIS will map data and provide a way for users to understand the future stranding of sea creatures. This will create a system to allow researchers and conservationists alike to better understand the sea creature beaching problem. This will help prevent sea creature deaths.
    \end{abstract}
        
       \pagebreak

       \section{Problem}
          The primary goal of this project is to decrease the deaths of sea creatures do to beach stranding. The expectations are that the group will create a better infrastructure for researchers to understand sea creature stranding and what causes them. The easiest way to approach this problem is giving researchers a visual representation of the problem. As sea temperatures continue to rise and there become more and more issues in the marine environment. Researchers will continue to need better and better ways to describe the problem. There aren't any easy applications to use that will allow researchers around the globe to use that will give them an effective representation of the data. This problem is not native to the Oregon coast or even the pacific coast. This problem spreads worldwide with researchers all around the world trying to better understand what is causing these animals to be stranded aground at alarming rates.

  \section{Solution}
      The solution to this devastating problem is to provide rescue teams and researchers alike with a way to better understand the problem of sea creature stranding. This can be done by using the data collected to create a visual model for researchers to understand the data. With a visual model researchers will be able to better interpret the data. This will allow for a better prediction on how sea creatures get stranded and what causes them to be stranded. This visual representation will also allow rescuers to better understand where exactly the animal was reported. There is a gaping whole the the oceanologist community for a local system to report and provide data on sea creature stranding. There is also a need for a global visual system. Currently NOAA has a system imp lace to report animals but a visual system could be beneficial to all parties. This system will provide a new piece of technical software to the scientific community and will allow scietists a way to better understand the world around them. This piece of technology is hopefully a new piece of technology that NOAA, Hatfield, and other marine researchers will be able to use.
      
      \subsection{Technical solution}
      The technical solution to this problem would be introducing a GIS system to the collected information in the database. The team is looking into gathering the NOAA's database's information to better provide a national GIS mapping system for sea creature strandings. This would allow researchers to gather information from a global database instead of a localized one. There is a lot of ways to approach this problem. Unfortunately, the team has yet to meet with the group manager. This means that the team does not have a clear idea of what the background of the pre-existing application and website are. The client has yet to convey his interest into developing ontop of the pre-existing system or creating a new private application to attach to the website. The method the team was suggesting was to provide a new application that connected to the database. That way if the website needed to be reworked the program wouldn't be attached to the website. This could be offered through a download link on the website to better the services provided. This application would be created for windows 10 and would allow researchers to view the data visually. This could be developed in C plus plus using ArcGIS. However, ArcGIS is a paid subscription so there could be an issue of financial backing. This system would be used to create a GUI that is simple and usable for all. The system could possibly allow for narrowing results and providing specific stranding within a certain distance. Depending on which GIS system is used there could be a graphic displaying what type of animal and when it was stranded. There are many interesting solutions to this project. This is a global problem with a very simple and clear cut goal, to help researchers better understand sea creature stranding.
        
      \section{Criteria}
        This projects success will depend on the ability to provide it's users with a better understanding of sea creature stranding. Since the team is unsure of the practicality of developing a predictive model, it could be simple or incredibly complex. There must be a simple line drawn of if this project has failed or completed it's goals. This goal is to better understanding. This means that this team will develop a mapping system for sea creatures that have been stranded in the past. Researchers will be able to use that data to visually understand whats going on with these sea creatures. This will be developed within an application for oceanologists to use. This is a reasonable criteria considering that we've not met with our client to understand what he already has and doesn't have. The application with be functional with everything outlined in this section, as long as more knowledge isn't gained.
       


      \section{Conclusion}
          To conclude this project will exist to prevent the deaths of sea creatures. The project is a conservationists project and will allow students, researchers, and rescuers to better understand why and how beachings happen. This visual model for a problem that isn't commonly known is important to better understanding. The system will be used by many that create information and prevent strandings up and down the Oregon coast and possibly nationally. This has the potential to be an incredibly useful system to oceanographers.

        
        \end{document}
