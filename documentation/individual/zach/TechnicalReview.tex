\documentclass[onecolumn, draftclsnofoot,10pt, compsoc]{IEEEtran}
\usepackage{graphicx}
\usepackage{url}
\usepackage{setspace}
\usepackage{times}
\usepackage{enumitem}
\usepackage{geometry}
\geometry{textheight=9.5in, textwidth=7in}

% 1. Fill in these details
\def \CapstoneTeamName{     Beached Marine Critters Project Team}
\def \CapstoneTeamNumber{       64}
\def \GroupMemberOne{           Alea Weeks}
\def \GroupMemberTwo{           Amar Raad}
\def \GroupMemberThree{         Daniel Domme}
\def \GroupMemberFour{          Justin Disalvo}
\def \GroupMemberFive{          Zachary Tusing}
\def \CapstoneProjectName{      Develop a visual model for sea turtle beach strandings}
\def \CapstoneSponsorCompany{   Oregon State University Hatfield Marine Science Center; Oregon Sea Grant}
\def \CapstoneSponsorPerson{        Dr. William Hanshumaker}

% 2. Uncomment the appropriate line below so that the document type works
\def \DocType{      %Problem Statement
                %Requirements Document
                Technology Review
                %Design Document
                %Progress Report
                }
            
\newcommand{\NameSigPair}[1]{\par
\makebox[2.75in][r]{#1} \hfil   \makebox[3.25in]{\makebox[2.25in]{\hrulefill} \hfill        \makebox[.75in]{\hrulefill}}
\par\vspace{-12pt} \textit{\tiny\noindent
\makebox[2.75in]{} \hfil        \makebox[3.25in]{\makebox[2.25in][r]{Signature} \hfill  \makebox[.75in][r]{Date}}}}
% 3. If the document is not to be signed, uncomment the RENEWcommand below
\renewcommand{\NameSigPair}[1]{#1}


            

%%%%%%%%%%%%%%%%%%%%%%%%%%%%%%%%%%%%%%%
\begin{document}
\begin{titlepage}
    \pagenumbering{gobble}
    \begin{singlespace}
     \includegraphics[height=3cm]{coe_v_spot1}
        \hfill 
        % 4. If you have a logo, use this includegraphics command to put it on the coversheet.
        %\includegraphics[height=4cm]{CompanyLogo}   
        \par\vspace{.2in}
        \centering
        \scshape{
            \huge CS Capstone \DocType \par
            {\normalsize\today}\par
            \vspace{.5in}
            \textbf{\Huge\CapstoneProjectName}\par
            %\vfill
            \vspace{1in}
            {\Large Prepared for}\par
            \huge \CapstoneSponsorCompany\par
            \vspace{5pt}
            {\Large\NameSigPair{\CapstoneSponsorPerson}\par}
            \vspace{.5in}
            {\large Prepared by }\par
            Group\CapstoneTeamNumber\par
            % 5. comment out the line below this one if you do not wish to name your team
            %\CapstoneTeamName\par 
            \vspace{5pt}
            {\Large
            \NameSigPair{\GroupMemberFive}\par
            }
            \vspace{20pt}
        }
        \vfill    
    \end{singlespace}
\end{titlepage}
\newpage

        
      \pagebreak
        \section{Introduction}
    This documents purpose is to provide technical solutions to issues within the marine creature predictive model product. This will be done by an analysis of adding information overlays to the GIS system.
      \section{GIS Systems}
      \subsection{OS Geo}
        Firstly, there is a large issue when introducing GIS systems to the group. The largest barrier often being the amount of money it takes to keep a license for a library or software. This is one of the largest barriers when looking into the GIS space. This is easily counter able by OS Geo. This is a library for implementing GIS systems and data mapping in c++. This is one of the few libraries that offer a lot of different information and open source work. It has been noted that this library does struggle with mapping many variables in an easy way. This however can be countered by the accessibility of a c++ GUI implementation. The OS Geo will allow us to easily transition into programming our problem. This will be due to most of the students in our group only knowing cpp and not having a good grasp on any other languages. 
      \subsection{ArcGIS}
        When talking about a big price tag to come along with GIS software, arcGIS seems to have one of the largest ones from what I can find. The price tag for a named user of the system is 500. This tag is completely unaffordable for our project unless there is a educational background information that we do not know about. This system has been noted to be one of the best systems for graphic interpretation of data and to easily work with and implement into our code. This piece of software would probably make our lives the easiest but it comes with a large price tag. The notable price tag does offer a lot of benefits to using the system as a developer. With the system we will be allowed to use many different high quality functions that aren't offered with other systems.
      \subsection{Q GIS}
        Q GIS might be the sweet spot for our project. This piece of software allows for open source graphics data mapping. There is an open source licence that would allow us easy use of the product. This will allow the product to easily be incorperated into our system. QGIS also allows for python implementation and when you're using a program to data manipulation python is one of the easiest languages to use. This comes along with that our data set is reasonably small making there be less of a reason for us to want to look at a php and sql implementation of our database. Our dataset is small emough where reading it in from a file will only take a fraction of a second. QGIS seems to be the sweet spot for our project and my suggested GIS system. The PyQGIS library seems to have a lot of developement potential. There are many different pieces of documentation provided to developers. This often means that it is one of the most highly recommended software for developers. With the open source nature of the program it will allow us to look into the library itself and dig out any functions we will need to implement into our system. This seems to be the easiest system for creating an overlay out of all the GIS systems mentioned here.
      \section{GUI Libraries}
      \subsection{PyQt}
      PyQt is a GUI framework for python. This library allows for a lot of different systems. PyQt will run on all platforms and provides the users with alot more than just a GUI framework. The syste provides with an abstraction of network sockets, threads, and even SQL databases. The system works in syncronous of many other Python libraries and allows the easiest and simplest implementation of Python GUIs to date. There is a lot to be envied by the system that implements the cross platform CPP library QT with the compile at runtime Python. The ease of using PyQt comes from accessing data with many different functions of Pythons massive library system. Python allows for easy access to many different types of functions without the need to write and compile them into your own library. PyQt is probably one of the faster frameworks for GUI implementations when working in Python. There are generally a lot of time issues when running a GUI in python. There is a large documentation system that comes with PyQt and it offers a lot of cross platform implemenation that languages like Java generally only tend to offer. PyQt would be my suggested GUI for our project.
      \subsection{ImGui}
      ImGui is a open source C Plus PLus library allowing for easy use and implentation when working with CPP. CPP libraries often come with licenses and purchasing but this perticular library. This offers a heap of GUI functions for use in a CPP program. However, this system doesn't seem to be incredibly well documented. The open source nature of the library allows for easy use in many different programs and a deep search of the library will reveal everything it has to offer and how it needs to be used. However, this will take a lot of time in compairison to a library that offers a simple documentation approach with a heap of questions being asked about it. This could be mirriored with the QT library in CPP. There are many benifits to using a Python version of the QT library that outweight the CPP version. This is why I am choosing to offer up ImGui as the go to CPP option.
      \subsection{Tkinter}
      Tkinter is a Python implementation of a GUI library. The main reason to implement the project using Tkinter is the simplicity of the library or framework. This library allows for the simplest approach to using a GUI. The entire system is intuitive to the point where there isn't a ton of documentation provide. This does offer a lot of use when working with the Microsoft Visual Studio. The Tkinter library and it's keywords are already inserted into the IDE. This allows for a simple way to program and execute the code. However, the benefits of using an IDE outweight the drawbacks of a lack of documentation. The main issue with using Tkinter over a system such as PyQt is the level of documentation being presented. Tkinter seems to be an incredibly slow and undocumented library. Tkinter doesn't provide enough benefits to outweigh the use of PyQt in my personal opinion.

      \section{Programming languages}
      \subsection{Python}
      Python seems like a simple solution as the languge of choice of GIS systems. With scripting being done simple and quickly this allows for a quick and easy execution and data search for our problem. The python implementation for data translation will be incredibly quick. There could be possible programs when recieving data from NOAA or US Fish and Wildlife. This could result in recieving visual data instead of recieving hard charts or sql dumps. This could potentially mean that there is a need for data translation and when implementing a system like this it could be incredibly helpful to pick the easiest data translation system. A python implemntation would allow for a quick pickup on the knowledge of the language since a lot of students programmed in this language during the first year of college. Python is perticularly my go to language when working with data and manipulating it. However, Python doesn't offer a good system for GUI implementations. GUI implementations can work incredibly slowly and difficultly in Python due to it's compile at runtime nature. This can mean a lot of slow boots in the program and possible problems with that.
      \subsection{CPP}
      C Plus Plus is probably the safest language for most programmers coming out of Oregon State. This is due to the excess of time spent programming in base C and CPP. With all of the knowledge and experience this would obviously be the easiest choice for our program to be implemented in. This could be a possible mesh of CPP and python allowing for easy data manipulation while also providing a good GUI and runtime speed. This might be the simplest way to implement a system for manipulating and reading data sheets. The issue with CPP is there is no easy way to read data effectively as there is in Python. Often the data needs to be incredibly formulaic in order to not cause problems when storing it. This could be fixed by vectors but my inexperience with CPP vectors means that I am weary of introducing a system of this magnitude with them. There is a lot that CPP can do for the project to provide a simpler solution to a lot of the GUI work that Python will struggle to handle.
      \subsection{Java}
      Java is finally the end all backup for cross platform implementations. When programming in Java you can always expect two things, ok performance, and ok look. Java is the king at bringing the ease to any system. When looking into different programming languages Java is always the simplest solution to any problem. There is little Java can't do but it rarely does it efficiently. The Java framework is generally good enough to get most of the systems done but without efficiency. Since the efficiency of this project doesn't really matter all that much Java could possibly be the perfect platform. There is little to nothing Java can't implement and the programming language is incredibly easy to pick up. 

      \section{Conclusion}
      There are many more systems that could be used than just the ones suggested here. The options discussed lead the direction of the project towards a dual-language implementation. This would be done by mixing the C++ GUI uses with the python GIS systems. This would be the most effective way to approach this problem. However, this could lead to some difficulties depending on the teams ablilities. The suggestion is to use a dual implementation as long as the team feels comfortable doing so.
        \end{document}
      