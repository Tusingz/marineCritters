\documentclass[onecolumn, draftclsnofoot,10pt, compsoc]{IEEEtran}
\usepackage{graphicx}
\usepackage{url}
\usepackage{setspace}
\usepackage{times}
\usepackage{enumitem}
\usepackage{geometry}
\usepackage{cite}

\geometry{textheight=9.5in, textwidth=7in}

% 1. Fill in these details
\def \CapstoneTeamName{		Beached Marine Critters Project Team}
\def \CapstoneTeamNumber{		64}
\def \GroupMemberOne{			Alea Weeks}
\def \GroupMemberTwo{			Amar Raad}
\def \GroupMemberThree{			Daniel Domme}
\def \GroupMemberFour{			Justin Disalvo}
\def \GroupMemberFive{			Zachary Tusing}
\def \CapstoneProjectName{		Developing a visual model for sea turtle beach strandings}
\def \CapstoneSponsorCompany{	Oregon State University Hatfield Marine Science Center; Oregon Sea Grant}
\def \CapstoneSponsorPerson{		Dr. William Hanshumaker}
\def \GroupRole{Archival Data Gatherer and Data Sorting Algorithm Handler}

% 2. Uncomment the appropriate line below so that the document type works
\def \DocType{		%Problem Statement
				%Requirements Document
				Technology Review
				%Design Document
				%Progress Report
				}
			
\newcommand{\NameSigPair}[1]{\par
\makebox[2.75in][r]{#1} \hfil 	\makebox[3.25in]{\makebox[2.25in]{\hrulefill} \hfill		\makebox[.75in]{\hrulefill}}
\par\vspace{-12pt} \textit{\tiny\noindent
\makebox[2.75in]{} \hfil		\makebox[3.25in]{\makebox[2.25in][r]{Signature} \hfill	\makebox[.75in][r]{Date}}}}
% 3. If the document is not to be signed, uncomment the RENEWcommand below
\renewcommand{\NameSigPair}[1]{#1}

%%%%%%%%%%%%%%%%%%%%%%%%%%%%%%%%%%%%%%%
\begin{document}
\begin{titlepage}
    \pagenumbering{gobble}
    \begin{singlespace}
     \includegraphics[height=3cm]{coe_v_spot1}
        \hfill 
        % 4. If you have a logo, use this includegraphics command to put it on the coversheet.
        %\includegraphics[height=4cm]{CompanyLogo}   
        \par\vspace{.2in}
        \centering
        \scshape{
            \huge CS Capstone \DocType \par
            {\normalsize\today}\par
            \vspace{.5in}
            \textbf{\Huge\CapstoneProjectName}\par
            %\vfill
            \vspace{1in}
            {\Large Prepared for}\par
            \huge \CapstoneSponsorCompany\par
            \vspace{5pt}
            {\Large\NameSigPair{\CapstoneSponsorPerson}\par}
            \vspace{.5in}
            {\large Prepared by }\par
            Group\CapstoneTeamNumber\par
            % 5. comment out the line below this one if you do not wish to name your team
            %\CapstoneTeamName\par 
            \vspace{5pt}
            {\Large
                %\NameSigPair{\GroupMemberOne}\par
                %\NameSigPair{\GroupMemberTwo}\par
                \NameSigPair{\GroupMemberThree} -
				\GroupRole\par
				%\NameSigPair{\GroupMemberFour}\par
			%\NameSigPair{\GroupMemberFive}\par
            }
            \vspace{20pt}
        }
        \vfill
        \begin{abstract}
        % 6. Fill in your abstract    
        	%This document is written using one sentence per line.
        	%This allows you to have sensible diffs when you use \LaTeX with version control, as well as giving a quick visual test to see if sentences are too short/long.
        	%If you have questions, ``The Not So Short Guide to LaTeX'' is a great resource (\url{https://tobi.oetiker.ch/lshort/lshort.pdf})
		    The Pacific Ocean off the Oregon Coast has constantly changing weather and sea conditions.  With this, many different animal species, including threatened sea turtles, end up stranded
            on the shore and perishing. Our group will to make an application to help study and better understand how weather and ocean conditions affect where and when animals get stranded. 
			Possible sources of historical data on sea turtles in the Pacific Northwest will be examined, and historical trends that examine the relation between climate and animal migration will be researched
			for possible use later on to determine what data might be relevant to the issue of animals becoming stranded.  Finally, user experience research will be examined to determine how to best organize 
			and display information and functions for the users of the software.
        \end{abstract}     
    \end{singlespace}
\end{titlepage}
\newpage
\pagenumbering{arabic}
\tableofcontents
% 7. uncomment this (if applicable). Consider adding a page break.
%\listoffigures
%\listoftables
\clearpage

% 8. now you write!
\begin{singlespace}
\section{Introduction}
This tech review will explore several sources of data and several methods of organizing and
analyzing that data to help our group meet our client's needs. I will be looking at research
organizations and government agencies that might be able to furnish us with historical data
on sea turtles in our area. Then, I will explore research that has been done on the effects of
sea and weather conditions on turtles. Finally, I will compare different sorting algorithms
that our project might use once our data is gathered.
\section{Data Sources}
Our project heavily relies on having access to a large amount of accurate and detailed
archival data on the stranding of sea turtles. Limiting the geographical scope of data to
the Oregon Coast and focusing on one species of animal means that our sources of data
need to be invested in our narrow requirements. The fact that sea turtle conservation and
climate change is a big issue in the world, will help to make our task easier. Federal
grants and funding have also helped in this field. In addition to this, our application is
meant to aid any organization or individual directly involved in the rescue, study, and
rehabilitation of sea turtles.
\subsubsection{U.S. Fish and Wildlife Service in Oregon}
The U.S. Fish and Wildlife Service (FWS) has multiple field offices throughout the state
of Oregon. The field office that might be of use to this project is located on the Oregon
Coast in Newport, Oregon. Since the FWS is a federal agency that is tasked with
managing and preserving wildlife and provides grants for research and conservation
activities, it seems like a logical organization to turn to for data on sea turtles getting stranded.
Being a federal agency also means that it will likely share information with the public,
and it requires grant holders to report findings for accountability. The Newport FWS
field office website points to the fact that it is responsible for beach all along the Oregon
Coast, and it encourages getting involved in conservation activities along with
contacting the office. \cite{ofwo}
\subsubsection{Oregon Marine Mammal Stranding Network}
Oregon State University's (OSU) Oregon Marine Mammal Stranding Network operates
in Newport Oregon at the Hatfield Marine Science Center. The stated purpose of this
program is to ``promote scientific investigation of marine mammal stranding events'' \cite{ommsn}.
The program's purpose aligns directly to our project's purpose. Advice is provided for
individuals to report stranded animals, and there are maps on the website that display
recent reports of animal sightings. Because the program's goal aligns directly with our
project and the program is part of OSU, we should have a good chance at having a
successful exchange of information. The data would also likely be in a format that
would lend itself to easy implementation of our software. Additionally, our program
might be used to expand upon their existing mapping of stranded animals.\cite{ommsn}
\subsubsection{Northwest Association of Networked Ocean Observing Systems (NANOOS)}
NANOOS is a nonprofit research organization that focuses primarily on the Oregon and
Washington Coasts. The research that they conduct appears to be relevant to our project. 
They may be able to provide us with data on sea and weather conditions and on animals
in the area. If there is no direct data on sea turtles, the organization may be able to point
our group to new sources of information. Their site has many different types and
displays of data, and the website is what our client directed our group to for an idea of a
finished product. NANOOS may be able to furnish us with details on how to implement
a predictive model and third-party libraries to use in our software. Oregon State
University is already a member of the organization, so this might help us get access to
information and form a mutually beneficial relationship.\cite{nanoos}
\section{Climate and Sea Animal Interaction Studies}
To save time and help in the success of our project, we might want to look to peer-reviewed
research papers. Hints and correlations as to what weather and sea events that might lead to
marine animals being stranded on beaches might have already been discovered and
published. By examining scientific articles on the topic of sea turtles and climate,
we might not need to invest as much time on finding correlations between conditions and outcomes.
Knowledge gained about turtle behavior in various situations should also provide some
insight in to possible outcomes. A few possible sources are presented below that will provide
context and knowledge to our project.
\subsubsection{Probability of Sea Turtle Stranding}
The 2018 article ``Likely locations of sea turtle stranding mortality using experimentally-calibrated,
time and space-specific drift models'' by Santos, et al. \cite{SantosBiancaS.2018Llos} tackles, exactly, what
our project is attempting to do: develop a predictive model of sea turtle stranding. This
article should be very useful to study the authors’ findings and apply them to our project.
Although the article focuses on predicting stranding on Virginia beaches and looks at sea
traffic effects on dead sea turtles, the topic is relatively the same. They take a set of data
and look at conditions to try to understand where the next sea turtle will land ashore.
This paper could be integral to our project, even if methods and geography do not match.
\subsubsection{Modelling of Effects of Climate}
One possible variable that might affect how sea turtles behave is temperature. This topic
is explored in Noga Neeman's 2014 dissertation called ``Mechanistic modeling of the
effects of climate change on sea turtle migration to nesting beaches'' \cite{NeemanNoga2014Mmot}. In Neeman's
dissertation, a model is developed to predict sea turtle behavior based on temperature
change. It discusses migration and remigration of turtles and how food is affected. The
paper stresses the need of being able to predict the future populations and behaviors to
better protect any at-risk species. This information will be useful in understanding
where and when to look on the Oregon Coast if there are any events where temperatures
are abnormal. Furthermore, it will help us to apply our model more thoroughly to other
parts of the world. We might be able to see a sea temperature change in another part of
the world and know where turtles might end up.
\subsubsection{The Effects of El Ni{\~{n}}o on Sea Turtles}
One topic that came up multiple times while talking with our client, Dr. William
Hanshumaker, was climate aberrations. Specifically, the El Ni{\~{n}}o and La Ni{\~{n}}a weather
phenomena in the Pacific Oceans. The question about what effect these events might
have on how sea turtles' behavior and outcomes will be something that our project might
need to look at to make an effective predictive model. The 2007 article ``The effect of
the El Ni{\~{n}}o Southern Oscillation on the reproductive frequency of eastern Pacific
leatherback turtles'' by Saba, et al. \cite{SabaVincentS2007Teot} could help to shed some light on this area. The topic
of El Ni{\~{n}}o is still being studied, and the effects of it are not well known on many
animals. By focusing the study of effects closer to the equator, conclusions might be
able to be drawn that could apply to the Oregon Coast as well. Sea water current and
temperature changes could affect where food is located or where the turtles are pushed.
This could help explain why some of these animals are present in an abnormal area (the
Oregon Coast).
\section{Sorting of Data}
With the inclusion of outside data in our project, we will need to deal with its importation
and organization into our software product. Once the data has been imported, it might need
to be resorted based on a different quality contained in each element. Having an inefficient
sorting algorithm for large amounts of data could cause our project to be hard to use at best
and nonfunctional at worst. We would like for our software product to be resilient and
reliable well into the future. This means that we should design our product with the inclusion
of large datasets in mind, even if our current data is small. We will need to make sure that
our program does not need large amounts of computing resources to sort any data because we
will not know what limitations user devices might have. In the following subsections, I will
explore several options to consider for inclusion into our project software.

\subsubsection{Merge Sort}
Merge sort is a relatively efficient algorithm for sorting large amounts of data. It is a
divide and conquer algorithm, which means that it splits up the task of sorting and
comparing into smaller problems. This makes it fast in almost all cases. Merge sort is
$O(n\log n)$ for best, worst, and average cases. This means that the time to sort any size
list of data will be slightly worse than linear. The algorithm will still be relatively fast
for large datasets. It is very consistent and efficient, which makes it a safe bet to use for
our project. The space complexity is $O(n)$, which is also good. This means that sorting
large amounts of data won’t take up unnecessary amount of space on hardware to
complete the computation of sorting everything in order. The only problems with merge
sort are the fact that it is recursive and that there are no linear or constant time
complexities. Recursion can take up a lot of resources, unfortunately. Other algorithms
may offer better time complexities for different data organizations, but they do not offer
the consistency that merge sort offers. This will probably be our method of sorting data
for our project.\cite{PairaSmita2016EMSA}
\subsubsection{Quick Sort}
Quick sort is one of the most common sorting algorithms used today. It is easy to
implement, and it can offer great speed and efficiency. It uses divide and conquer to
help simplify the task of sorting. First, one element of data is picked that all others will be 
compared against. Then, it takes the front of a list of data and the last and finds the
middle. The algorithm recursively does this again repeatedly until only two elements are
left, then compared. If the elements need to be swapped, swap them. This is done with
all other elements, and eventually, a sorted list of elements is returned. The best case for
use of quick sort on a set of data is a time complexity of $O(n)$. This means that all
elements are sorted in linear time. This is very good complexity. The average case is
$O(n\log n)$. This is still good. It falls just short of linear time complexity. The worst
case is $O(n^{2})$. This is bad and means that, if there was a large data set organized in the
worst-case way, the time to sort would be quite long. The space complexity is, at worst,
$O(\log n)$, which is not bad. This algorithm might be useful for us, although it can be bad
on time. \cite{SkienaStevenS2008TADM}
\subsubsection{Insertion Sort}
Insertion sort appears to be a less efficient sorting algorithm. Data is sorted by iterating
over every item and seeing if it needs to be swapped with another. It has an inner and
outer loop that, at worst, will run twice, which means that the worst-case time
complexity is $O(n^{2})$. The average case is also not great: $O(n^{2})$. This means that, most
of the time, the time that insertion sort takes to sort an amount of data is increasing at a
large rate as the size of data that needs to be sorted increases. This does not make the
algorithm ideal for our project's goals. The only advantages are that the best case has a
linear time complexity of $O(n)$ and the space complexity is constant. Our group will
probably not use this method for sorting our data in our project.\cite{CrisanDanielaAlexandra2015Rafs}
\section{Conclusion}
This document has looked at multiple topics that will help our group to complete its software project.  
First, potential sources of archived data on sea turtles were discussed.  
They will be needed to be able to make predictions on animals being stranded.  Hopefully, each of the listed sources will be used.  
Then, journal articles that might be able to give us ideas on variables to consider when analyzing data were examined.  
The use of all three journals might also help us reach stretch goals.  Finally, three different sorting algorithms and their pros and cons were weighed.  
Due to its efficiency, merge sort appears to be the algorithm that will be implemented to organize data.  
All of these topics and content of this tech review will be instrumental to the success of our project, and it will aid in rescue and study marine animals.
\newpage
\bibliography{sources}
\bibliographystyle{IEEEtran}


\end{singlespace}
\end{document}